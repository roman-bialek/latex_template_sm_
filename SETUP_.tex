% Setup file for custom commands
% ------------------------------------------------------------
% Packages: geometry and utils
% \usepackage{geometry}
\usepackage{blindtext}
\usepackage{graphicx}
\usepackage{pgfplots}
\pgfplotsset{compat = newest}%\pgfplotsset{compat=1.18} 

% ------------------------------------------------------------
% Packages: Mathematics
\usepackage{hyperref}
\usepackage{mathtools}
\usepackage{amsmath, bm}
\usepackage{esint} % enables closed surface integral
\usepackage{amsthm}
\usepackage{amsfonts}
\newcommand{\defas}{\coloneqq{}}

% ------------------------------------------------------------
%   Mathematics custom
% > Dimensional UNITS ~ ~ ~ ~ ~ ~ ~ ~ ~ ~ ~ ~ ~ ~ ~ ~ ~ ~ ~ ~ 
\newcommand{\units}[1]
{
	\left[ #1 \right]
}

\newcommand{\metre}{\mrm{m}}

\newcommand{\second}{\mrm{s}}

\newcommand{\kelvin}{\mrm{K}}

\newcommand{\kg}{\mrm{kg}}
\newcommand{\gram}{\mrm{g}}

\newcommand{\mol}{\mrm{mol}}

\newcommand{\amps}{\mrm{A}}

\newcommand{\pascal}{\mrm{Pa}}

\newcommand{\coulomb}{\mrm{C}}
\newcommand{\farad}{\mrm{F}}

\DeclareMathOperator{\UnitsOf}{units}


% > Math Operator Formatting  ~ ~ ~ ~ ~ ~ ~ ~ ~ ~ ~ ~ ~ ~ ~ ~ 
\newcommand{\algt}{ \, } % alg[bra] t[imes]

\newcommand{\transpose}{\mathsf{T}}
\newcommand{\cross}{\times{}}

% Differential operators
\newcommand{\de}[0]{\mathrm{d}}
\newcommand{\dd}[2]{\frac{\mathrm{d}{#1}}{\mathrm{d} {#2}}}
\newcommand{\ddn}[3]{\frac{\mathrm{d}^{#3}{#1}}{\mathrm{d} {#2}^{#3}}}

% Partial differential commands
\newcommand{\pd}[2]{\frac{\partial{}{#1}}{\partial{} {#2}}}
\newcommand{\pdbrac}[2]{\frac{\partial{}}{\partial{} {#2}} \left( {#1} \right)}

\newcommand{\pdn}[3]{\frac{\partial{}^{#3}{#1}}{\partial{} {#2}^{#3}}}
\newcommand{\pdnbrac}[3]{\frac{\partial{}^{#3}}{\partial{} {#2}^{#3}} \left( {#1} \right)}


% Total derivative operators
\newcommand{\TD}[2]{\frac{\mathrm{D}{#1}}{\mathrm{D} {#2}}}
\newcommand{\TDN}[3]{\frac{\mathrm{D}^{#3} {#1}}{\mathrm{D} {#2} ^{#3}}}

% Mean Formatter
\newcommand{\mean}[1]{\overline{#1}}


% > Math Integration ~ ~ ~ ~ ~ ~ ~ ~ ~ ~ ~ ~ ~ ~ ~ ~ ~ ~ ~ ~ ~
\newcommand{\surfint}[2]{\int_{\partial{}#1} {#2} \; \mrm{d}S}
\newcommand{\volint}[2]{\int_{#1} {#2} \; \mrm{d}\mathcal{V}}

% provides `D`irect underscore (\partial-less)
\newcommand{\surfintD}[2]{\int_{#1} {#2} \; \mrm{d}S}


% > Math Vector Basis text format ~ ~ ~ ~ ~ ~ ~ ~ ~ ~ ~ ~ ~ ~ 
\newcommand{\eb}[1]
{
	\bm{e}_{#1}
}

\renewcommand{\vec}[1]
{
	\bm{#1}
}

\newcommand{\tensor}[1]
{
	\bm{#1}
}

\newcommand{\vecrm}[1]
{
	\mathbf{#1}
}

\newcommand{\veci}[2]
{
	{#1}_{#2}
}

% This one is pointless
%\newcommand{\tensor}[3]
%{
	%	{#1}_{#2}^{#3}
	%}

\newcommand{\vecbasis}[2]
{
	\bm{#1}_{#2}
}
\newcommand{\vecbasisrm}[2]
{
	\mathbf{#1}_{#2}
}

\newcommand{\levicivita}{\epsilon{}}

% > [New mod] div curl etc ~ ~ ~ ~ ~ ~ ~ ~ ~ ~ ~ ~ ~ ~ 
\newcommand{\Div}[0]
{
	\nabla \cdot 
}

\newcommand{\Grad}[0]
{
	\nabla \,
}

\newcommand{\Curl}[0]
{
	\nabla \cross
}

\newcommand{\Laplacian}[0]
{
	\nabla^2 \,
}

\newcommand{\Divrm}[0]
{
	\mrm{div} \,
}

% > Math Text Formatting  ~ ~ ~ ~ ~ ~ ~ ~ ~ ~ ~ ~ ~ ~ ~ ~ ~ ~ 
\newcommand{\mrm}[1]
{
	\mathrm{#1}
}

\newcommand{\ndim}[1]
{
	{#1}^{*}
}
% > Math Bracket wrappers   ~ ~ ~ ~ ~ ~ ~ ~ ~ ~ ~ ~ ~ ~ ~ ~ ~ 
\newcommand{\LRBrac}[1]
{
	\left( #1 \right)
}

\newcommand{\lrbrac}[1]
{
	\left( #1 \right)
}

\newcommand{\lrbracS}[1]
{
	\left[ #1 \right]
}

\newcommand{\lrbracB}[1]
{
	\left\{ #1 \right\}
}

% ------------------------------------------------------------
% Custom amsthm assumptions
\newtheorem{theorem}{Theorem}[section]
\newtheorem{assumption}{Assumption}[section]
\newtheorem{concept}{Concept}[section]
\newtheorem{theory}{Theory}[section]

\theoremstyle{definition}
\newtheorem{definition}{Definition}[section]


\newtheorem{comment}{Comment}[section]
\newtheorem{revise}{Revise}[section]

\newtheorem{defi}{Definition}[]
\newtheorem{identity}{Identity}[]

% ------------------------------------------------------------
%  Figures and citation

% Figure custom captioning etc.
\usepackage[labelfont=bf,
figurename=Fig.,
labelsep=space]{caption} 

% use this for sub-figures
\usepackage{subcaption}

%% More packages
%\usepackage{tgtermes} % times font

\usepackage{cite}

\newcommand{\figref}[1]{Fig. \ref{#1}}

% ------------------------------------------------------------
% Code blocks
\usepackage{listings}
\usepackage{xcolor}

\definecolor{codegreen}{rgb}{0,0.6,0}
\definecolor{codegray}{rgb}{0.5,0.5,0.5}
\definecolor{codepurple}{rgb}{0.58,0,0.82}
\definecolor{backcolour}{rgb}{0.95,0.95,0.92}

\lstdefinestyle{mystyle}{
	backgroundcolor=\color{backcolour},   
	commentstyle=\color{codegreen},
	keywordstyle=\color{magenta},
	numberstyle=\tiny\color{codegray},
	stringstyle=\color{codepurple},
	basicstyle=\ttfamily\footnotesize,
	breakatwhitespace=false,         
	breaklines=true,                 
	captionpos=b,                    
	keepspaces=true,                 
	numbers=left,                    
	numbersep=5pt,                  
	showspaces=false,                
	showstringspaces=false,
	showtabs=false,                  
	tabsize=2
}

\lstdefinestyle{C}{
	backgroundcolor=\color{backcolour},   
	commentstyle=\color{codegreen},
	keywordstyle=\color{magenta},
	numberstyle=\tiny\color{codegray},
	stringstyle=\color{codepurple},
	basicstyle=\ttfamily\footnotesize,
	breakatwhitespace=false,         
	breaklines=true,                 
	captionpos=b,                    
	keepspaces=true,                 
	numbers=left,                    
	numbersep=5pt,                  
	showspaces=false,                
	showstringspaces=false,
	showtabs=false,                  
	tabsize=2,
	language=C
}

% ------------------------------------------------------------
% Use VERBATIM for inline code
\usepackage{verbatim}

% ------------------------------------------------------------
% `placeins` provides `\FloatBarrier{}` for figures/images
\usepackage{placeins}

% ------------------------------------------------------------
% Enable strike out, underlines, etc.
\usepackage{soul}

% ------------------------------------------------------------
% Enables 1000th's space separators, e.g., 1 000 240.001 150
\usepackage{numprint}

% ------------------------------------------------------------
% Typesetting logos / key words
\newcommand{\openfoam}{OpenFOAM}
\newcommand{\C}{\texttt{C}}
\newcommand{\CPP}{\texttt{C++}}
\newcommand{\OOP}{\texttt{OOP}}

% ------------------------------------------------------------
% At end
\usepackage{enumitem}
